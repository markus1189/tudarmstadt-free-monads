\documentclass{beamer}

\usepackage{minted}

\usetheme{codecentric}

\renewcommand*{\insertdate}{January 26, 2017}

\author{Markus Hauck}
\institute{codecentric AG}

\title{Warm fluffy Interpreters}

\begin{document}
{
  \usebackgroundtemplate{\includegraphics[width=\paperwidth,height=\paperheight]{background.png}}
  \begin{frame}[plain,noframenumbering]
    \titlepage{}
  \end{frame}
}

\begin{frame}
  \tableofcontents
\end{frame}


\section{Introduction}

\begin{frame}
  \frametitle{The codecentric beamer theme}
  \begin{itemize}
  \item this is a template for LaTeX beamer
  \item can be used to get started with a presentation
  \item the theme mimics codecentric style
  \item prs welcome!
  \end{itemize}
  \begin{enumerate}
  \item one
  \item two
  \item three
  \end{enumerate}
\end{frame}

\section{Typeclasses}

\begin{frame}
  \frametitle{Typeclasses}
  \begin{itemize}
  \item Design pattern coming from Haskell
  \item Another form of ad-hoc polymorphism (Quiz?)
  \item Can be encoded in Scala
  \end{itemize}
\end{frame}

\begin{frame}
  \frametitle{Typeclasses in Scala}
  \begin{enumerate}
  \item trait with at least one type parameter defining the available
    methods
  \item implicit value/def on supported types that define the
    behaviour
  \item syntactic sugar via implicit class
  \end{enumerate}
\end{frame}

\begin{frame}[fragile]
  \frametitle{The Monoid Typeclass: Trait}
\begin{minted}{scala}
trait Monoid[A] {
  def empty: A
  def combine(lhs: A, rhs: A): A
}
\end{minted}
\end{frame}

\begin{frame}[fragile]
  \frametitle{The Monoid Typeclass: Instance}
\begin{minted}{scala}
object Monoid {
  def apply[A:Monoid]: Monoid[A] = implicitly

  implicit val plusMonoid: Monoid[Int] = new Monoid {
    def empty: Int = 0
    def combine(lhs: Int, rhs: Int): Int = 0
  }
}
\end{minted}
\end{frame}

\begin{frame}[fragile]
  \frametitle{The Monoid Typeclass: Sugar}
\begin{minted}{scala}
implicit class MonoidOps[A](self: A)(
  implicit M: Monoid[A]
) {
  def |+|(that: A): A = M.combine(self,that)
}
\end{minted}
\end{frame}

\begin{frame}[fragile]
  \frametitle{The Monoid Typeclass: Usage}
\begin{minted}{scala}
scala> import Monoid._
scala> 1 |+| 2
res0: Int = 3
scala> Monoid[Int].empty
res1: Int = 0
scala> Monoid[Int].empty |+| 42
res2: Int = 42
\end{minted}
\end{frame}

\section{Monads}

\begin{frame}
  \frametitle{Monads}
  \begin{quote}
    Simon Peyton Jones once commented that the biggest mistake
    Haskell made was to call them ``monads'' instead of ``warm, fluffy
    things''.
  \end{quote}
\end{frame}

\section{Free Monads \& Interpreters}

\begin{frame}
  \frametitle{Free Monads}
  \begin{itemize}
  \item<1> Idea: strip away any semantics, only keep the AST
  \item<1> Quiz: what nodes do we have for monads? (what operations?)
  \item<2> pure \& flatMap
  \end{itemize}
\end{frame}

\begin{frame}
  \frametitle{Essence of Free Monads}
  \begin{itemize}
  \item build the AST via normal map/flatMap/pure operations
  \item \textbf{interpret} the resulting program via interpreter
  \end{itemize}
\end{frame}

\begin{frame}
  \frametitle{Applying Free Monads}
  \begin{itemize}
  \item Szenario: Our program interacts with Database
  \item<1> for testing we want to stub it
  \item<1> existing solutions: embedded version, in-memory etc.
  \item<2> in some cases still undesirable:
    \begin{itemize}
    \item embedded version requires substantial setup time
    \item makes parallel running tests very resource intensive
    \end{itemize}
  \item<2> better: base your operations on free monads, use different
    interpreters for production and test
  \end{itemize}
\end{frame}

\begin{frame}[fragile]
  \frametitle{Our Program}
\begin{minted}{scala}
def addItem(itemId: ItemId, uid: UserId) = for {
  item <- loadItem(itemId)
  cart <- loadCart(uid)
  newCart = cart.add(item)
  r <- saveCart(newCart)
} yield r
\end{minted}
\end{frame}

\section{Conclusion}

\begin{frame}
  \frametitle{Conclusion}
  \begin{description}
  \item[First description] This concludes the example presentation
  \item[Second description] Go forth and create beautiful presentations!
  \end{description}
\end{frame}

\end{document}
